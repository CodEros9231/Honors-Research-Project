\documentclass[12pt]{article}

% Adjust page margins
\usepackage[margin=1in]{geometry}

% Set font
\usepackage{times}

% Section style
\usepackage{titlesec}
\titleformat{\section}[hang]{\bfseries}{\thesection.}{1em}{}

% Set line spacing
\usepackage{setspace}
\doublespacing

\begin{document}

% Title
\begin{center}
\textbf{\LARGE Project Title}
\end{center}

% Faculty mentor
\begin{flushright}
Faculty Mentor \\
Department of Computer Engineering \\
University of XYZ
\end{flushright}

% Horizontal line
\noindent\rule{\textwidth}{0.4pt}

% Research question
\section{Research Question}
How can machine learning algorithms be optimized for the development of a handheld ultrasound device with refined ultrasound images, and what are the feasibility and potential applications of such a device?

% Background
\section{Background}
In recent years, there has been an increasing demand for portable medical devices that can provide accurate diagnosis and treatment in a timely and cost-effective manner. With the advancements in machine learning and artificial intelligence, there has been a growing interest in using these technologies to develop such devices for a range of medical applications.

Our proposed research aims to develop a handheld ultrasound device that utilizes deep neural networks to refine ultrasound images and provide accurate diagnosis of medical conditions. The device will be designed to be portable, cost-effective, and easy to use, making it ideal for use in remote areas or in emergency situations where time is of the essence.

The potential impact of this research is significant, as it has the potential to improve healthcare outcomes for millions of people worldwide, especially those living in underserved or rural areas. By providing accurate diagnosis and treatment at the point of care, this device has the potential to save lives, reduce healthcare costs, and improve patient outcomes


% Required skills
\section{Required Skills}
\begin{enumerate}
\item Proficiency in electrical engineering and circuit design
\item Knowledge of machine learning algorithms and techniques
\item Familiarity with programming languages such as Python and MATLAB
\item Ability to conduct experimental research and analyze data
\item Proficiency in using lab equipment and tools for testing and prototyping
\item Attention to detail and ability to troubleshoot technical issues
\item Ability to work independently and manage time effectively
\end{enumerate}

% Methodology and Timeline
\section{Methodology and Timeline}
Week-by-week breakdown of the tasks to be performed, for all 14 weeks.

\begin{enumerate}
    \item Week 1: Familiarize yourself with the project and relevant literature. Meet with your supervisor to clarify goals and expectations.
    \item Week 2: Start learning about machine learning algorithms and how they can be used in biomedical applications.
    \item Week 3:  Explore available datasets and start testing different algorithms on them.
    \item Week 4: Continue testing different algorithms and analyze their results. Discuss with your supervisor about the best approach to take..
    \item Week 5: Begin to refine the algorithms and consider how they can be optimized for the device.
     \item Week 6: Evaluate the effectiveness of the optimized algorithms and compare them to the previous results.
      \item Week 7: Begin to integrate the algorithms with the device and test their functionality.
       \item Week 8: Identify any problems or limitations in the integration process and work to resolve them.
        \item Week 9:Continue testing and refining the integrated device and algorithms.
\item Week 10: Analyze the performance of the integrated device and make improvements as needed.
\item Week 11: Develop a plan for the PCB design and begin creating the schematic.


         \item Week 12: Complete the PCB layout and send it for fabrication.

          \item Week 13: Assemble the PCB and test its functionality with the integrated device and algorithms.
           \item Week 14: Finalize the project and present your findings and prototype to your supervisor.

\end{enumerate}

% Project Output
\section{Project Output}
The expected output of this 14-week project is to refine and optimize the machine learning algorithms for handheld ultrasound imaging using deep neural networks. These algorithms will be implemented and tested on an already available device. The ideal outcome of the project is to successfully prototype a handheld ultrasound device with refined ultrasound images produced through deep neural networks.


% Grading
\section{Grading (to be completed by the faculty mentor)}
How will the Project Output be mapped to a grade, i.e. what are the expectations for an A, A-, B+, etc.

% Page break
\pagebreak

% Acknowledgements
\section*{Acknowledgements}
Acknowledgements, if any.

% References
\begin{thebibliography}{9}
\bibitem{example1}
  Example reference 1.
  
\bibitem{example2}
  Example reference 2.
\end{thebibliography}

\end{document}
